%!TEX encoding = UTF-8 Unicode
%!TEX root = ../lect-w11.tex

%%%


\Subsection{Veckans uppgifter: \texttt{scalajava} och \texttt{javatext}}
\begin{Slide}{Veckans uppgifter: \texttt{scalajava} och \texttt{javatext}}\SlideFontSmall
\Emph{Labbförberedelse:}
\begin{itemize}
\item Gör övning \texttt{scalajava}:
\begin{itemize}\SlideFontSmall
\item Översätt spelet Hangman från Java till Scala
\item Översätt Point från Scala till Java
\item Undersök autoboxning \Eng{autoboxing}
\item Använda \code{import scala.collection.JavaConverters._}
\end{itemize}
\item Studera riktlinjerna för \code{javatext} i kompendiet.
\item Labben är \Alert{individuell} men du ska \Emph{spela en tidig version av någon annans spel} och ge återkoppling på kodens \Alert{läsbarhet} och vice versa.
\end{itemize}
\Emph{Laboration \code{javatext}:}
\begin{itemize}
\item Gör klart ett (lagom) intressant/roligt textspel för terminalen med ca 80\% Java-kodrader och ca 20\% Scala-kodrader, enligt kraven i kompendiet.
\item \Alert{Relax}. För de som har flera "kompletteras" efter sig eller tycker
det är alldeles övermäktigt att få ihop de obligatoriska kraven är det
ok att skippa eller förenkla dessa krav: nr 2: spara på fil, nr 4: mäta tid.
\end{itemize}
\end{Slide}
